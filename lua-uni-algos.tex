\documentclass{article}
\usepackage{doc, shortvrb, metalogo, hyperref, fontspec}
% \setmainfont{Noto Serif}
% \setmonofont{FreeMono}
\title{Unicode algorithms for Lua\TeX}
\author{Marcel Krüger\thanks{E-Mail: \href{mailto:tex@2krueger.de}{\nolinkurl{tex@2krueger.de}}}}
\MakeShortVerb\|
\newcommand\pkg{\texttt}
\begin{document}
\maketitle
Dealing with general Unicode encoded data comes with many challenges because it has to respect individual concerns of many different scripts and languages. The Unicode consortium maintains multipple useful algorithms which can sometimes make this task much easier.

\pkg{lua-uni-algos} tries to make the most fundamental algorithms available for authors of Lua-based packages to aid in handling Unicode data.

Currently this package implements:
\begin{description}
  \item[Unicode normalization] Normalize a given Lua string into any of the normallization forms NFC, NFD, NFKC, or NFKD as specified in the Unicode standard, section 2.12.
  \item[Grapheme cluster segmentation] Identify a grapheme cluster, a unit of text which is perceived as a single character by typical users, according to the rules in UAX \#29, section 3.
\end{description}
\section{Normalization}
Unicode normalization is handled by the Lua module |lua-uni-normalize|.
You can either load it directly with
\begin{verbatim}
local normalize = require'lua-uni-normalize'
\end{verbatim}
or if you need access to all implemented algorithms you can use
\begin{verbatim}
local uni_algos = require'lua-uni-algos'
local normalize = uni_algos.normalize
\end{verbatim}

Then, four functions are available: |normalize.NFC|, |normalize.NFD|, |normalize.NFKC|, and |normalize.NFKD|.
If you do not know which of these you need, then you should probably |normalize.NFC|. All functions are used in the same way:
\begin{verbatim}
local str = "Äpfel…"
print("Original:", str)
print("NFC:", normalize.NFC(str))
print("NFD:", normalize.NFD(str))
print("NFKC:", normalize.NFKC(str))
print("NFKD:", normalize.NFKD(str))
\end{verbatim}
This results in
\begin{verbatim}
Original:	Äpfel…
NFC:	Äpfel…
NFD:	Äpfel…
NFKC:	Äpfel...
NFKD:	Äpfel...
\end{verbatim}
(This example is shown in Latin Modern Mono which has the (for this purpose) very useful property of not handling combining character very well.
In a well-behaving font, the `...C` and `...D` lines should look the same.)

\section{Grapheme clusters}
Grapheme cluster handling is handled by the Lua module |lua-uni-graphemes|.
You can either load it directly with
\begin{verbatim}
local graphemes = require'lua-uni-graphemes'
\end{verbatim}
or if you need access to all implemented algorithms you can use
\begin{verbatim}
local uni_algos = require'lua-uni-algos'
local graphemes = uni_algos.graphemes
\end{verbatim}

Sometimes we want to look at a single character of a string, but identifyig what a character is is not that easy in Unicode. A simple example is the character from the previous section: ``Ä''
The NFD form is certainly a single character, but is encoded using two codepoints: U+0041 (LATIN CAPITAL LETTER A) and U+0308 (COMBINING DIAERESIS). Or the Tamil letter Ni which is encoded as U+0BA8 (TAMIL LETTER NA) followed by U+0BBF (TAMIL VOWEL SIGN I). But sometimes it can be useful to identify characters, e.g.\ for letterspacing or letterines.

There are two main interfaces for this: One iterator for iterating over grapheme clusters and one direct interface to the underlying state machine:

\begin{verbatim}
for final, first, grapheme in graphemes.graphemes'Äpfel' do
  print(grapheme)
end
\end{verbatim}
% \begin{verbatim}
% for final, first, grapheme in graphemes.graphemes'Z͑ͫ̓ͪ̂ͫ̽͏̴̙̤̞͉͚̯̞̠͍A̴̵̜̰͔ͫ͗͢L̠ͨͧͩ͘G̴̻͈͍͔̹̑͗̎̅͛́Ǫ̵̹̻̝̳͂̌̌͘!͖̬̰̙̗̿̋ͥͥ̂ͣ̐́́͜͞' do
%   print(grapheme)
% end
% \end{verbatim}

\noindent\begingroup
  \ttfamily
  \directlua{
    local graphemes = require'./lua-uni-graphemes'
    for final, first, grapheme in graphemes.graphemes'Äpfel' do
      tex.sprint(grapheme, '\string\\\string\\')
    end
  }\par
\endgroup

The more powerful state machine interface |graphemes.read_codepoint| takes two parameters: A new codepoint and a state.
At the beginning, the state can be omitted.
For every codepoint in your input, call the function with the new codepoint and the last state. Then there are two return values: The first one is a boolean telling you if the current codepoint is the beginning of a new cluster, the second is a new state you have to pass with the next codepoint.

So e.g.\ to find cluster boundaries in the Unicode codepoint sequence U+0041 U+0308 U+0BA8 U+0BBF you could use

\begin{verbatim}
local graphemes = require'lua-uni-graphemes'
local new_cluster, state
new_cluster, state = graphemes.read_codepoint(0x0041, state)
print(new_cluster)
new_cluster, state = graphemes.read_codepoint(0x0308, state)
print(new_cluster)
new_cluster, state = graphemes.read_codepoint(0x0BA8, state)
print(new_cluster)
new_cluster, state = graphemes.read_codepoint(0x0BBF, state)
print(new_cluster)
\end{verbatim}
  
\noindent resulting in

\noindent\begingroup
  \ttfamily
  \directlua{
    local graphemes = require'lua-uni-graphemes'
    local new_cluster, state
    new_cluster, state = graphemes.read_codepoint(0x0041, state)
    tex.sprint(tostring(new_cluster), '\string\\\string\\')
    new_cluster, state = graphemes.read_codepoint(0x0308, state)
    tex.sprint(tostring(new_cluster), '\string\\\string\\')
    new_cluster, state = graphemes.read_codepoint(0x0BA8, state)
    tex.sprint(tostring(new_cluster), '\string\\\string\\')
    new_cluster, state = graphemes.read_codepoint(0x0BBF, state)
    tex.sprint(tostring(new_cluster), '\string\\\string\\')
  }\par
\endgroup

\vskip-\baselineskip
\noindent meaning the first and third codepoint start a new cluster.

Do not try to interpret the |state|, it has no defined values and might change at any point.

\end{document}
